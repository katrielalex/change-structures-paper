% Theorem styles 

\theoremstyle{plain}
\newtheorem{thm}{Theorem}
\newtheorem{lemma}[thm]{Lemma}

\theoremstyle{definition}
\newtheorem{prop}[thm]{Proposition}

\theoremstyle{remark}
\newtheorem{claim}[thm]{Claim}

\theoremstyle{remark}
\newtheorem{corollary}[thm]{Corollary}

\theoremstyle{remark}
\newtheorem{rem}[thm]{Remark}

% Definitions on their own counter

\theoremstyle{definition}
\newtheorem{defn}{Definition}

% Notation

% General
\newcommand{\defeq}{\coloneqq}
\newcommand{\cat}[1]{\mathbf{#1}}
\newcommand{\equalizer}[2]{Eq(#1, #2)}
\newcommand{\powerset}[1]{\mathcal{P}(#1)}
\newcommand{\denote}[1]{\llbracket #1 \rrbracket} 
\newcommand{\ev}{\operatorname{ev}} 
\newcommand{\doubleplus}{\ensuremath{\mathbin{+\mkern-10mu+}}}
\newcommand{\pair}[2]{\left\langle {#1}, {#2} \right\rangle}

% Change structures
% Generic "thing in a circle" operator
\makeatletter
\newcommand\cplussym
{
  \mathpalette\@incircbin
}
\newcommand\@incircbin[2]
{
  \mathbin
  {
    \ooalign{\hidewidth$#1#2$\hidewidth\crcr$#1\bigcirc$}%
  }
}
\makeatother

\newcommand{\cplus}{\oplus}
\newcommand{\cpluss}{\boxplus}
\newcommand{\cplusss}{\odot}
\newcommand{\cplusvee}{\cplussym{\vee}}
\newcommand{\cminus}{\ominus}

\newcommand{\splus}{\cdot}
\newcommand{\mzero}{\textbf{0}}

\newcommand{\cstruct}[3]{(#1,#2,#3)}
\newcommand{\changes}[1]{\Delta #1}
\newcommand{\change}[1]{\delta #1}

\newcommand{\discrete}{\emptyset}

\newcommand{\derive}[1]{#1'}
\newcommand{\supderive}[1]{#1_\uparrow}
\newcommand{\supderiveM}[1]{#1_{\uparrow\uparrow}}
\newcommand{\subderive}[1]{#1_\downarrow}
\newcommand{\subderiveM}[1]{#1_{\downarrow\downarrow}}

\newcommand{\monotoneDerive}[1]{#1'^{M}}

\newcommand{\difffunc}{\rightarrow_{D}}

% Algebra
\newcommand{\superpose}{\circledast}
\newcommand{\curry}[1]{\lambda #1}

% Orders
\newcommand{\reachOrder}{\leq_R}
\newcommand{\changeOrder}{\leq_\Delta}
\newcommand{\fineOrder}{\leq_D}
\newcommand{\minusOrder}{\leq_\cminus}

% Lattices
\newcommand{\twist}{\bowtie}
\newcommand{\updiff}{\Delta}
\newcommand{\downdiff}{\nabla}

% Fixpoints
\newcommand{\fixpoint}{\textbf{fix}}
\newcommand{\lfp}{\textbf{lfp}}

% Datalog
\newcommand{\Term}{\mathrm{Term}}
\newcommand{\Rel}{\cat{Rel}}
\newcommand{\universalRel}{\mathcal{U}}
\newcommand{\consq}{\mathcal{I}}