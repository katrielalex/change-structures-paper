\documentclass[english]{article}

\usepackage{hyperref}

\bibliographystyle{plain}

%%% For double-blind review submission, w/o CCS and ACM Reference (max submission space)
\documentclass[sigplan,10pt,review,anonymous,natbib=false]{acmart}\settopmatter{printfolios=true,printccs=false,printacmref=false}
%% For double-blind review submission, w/ CCS and ACM Reference
%\documentclass[sigplan,10pt,review,anonymous]{acmart}\settopmatter{printfolios=true}
%% For single-blind review submission, w/o CCS and ACM Reference (max submission space)
%\documentclass[sigplan,10pt,review]{acmart}\settopmatter{printfolios=true,printccs=false,printacmref=false}
%% For single-blind review submission, w/ CCS and ACM Reference
%\documentclass[sigplan,10pt,review]{acmart}\settopmatter{printfolios=true}
%% For final camera-ready submission, w/ required CCS and ACM Reference
%\documentclass[sigplan,10pt]{acmart}\settopmatter{}

%% Conference information
%% Supplied to authors by publisher for camera-ready submission;
%% use defaults for review submission.
\acmConference[PL'17]{ACM SIGPLAN Conference on Programming Languages}{January 01--03, 2017}{New York, NY, USA}
\acmYear{2017}
\acmISBN{} % \acmISBN{978-x-xxxx-xxxx-x/YY/MM}
\acmDOI{} % \acmDOI{10.1145/nnnnnnn.nnnnnnn}
\startPage{1}

%% Copyright information
%% Supplied to authors (based on authors' rights management selection;
%% see authors.acm.org) by publisher for camera-ready submission;
%% use 'none' for review submission.
\setcopyright{none}
%\setcopyright{acmcopyright}
%\setcopyright{acmlicensed}
%\setcopyright{rightsretained}
%\copyrightyear{2017}           %% If different from \acmYear

%% Bibliography style
\bibliographystyle{ACM-Reference-Format}
%% Citation style
%\citestyle{acmauthoryear}  %% For author/year citations
%\citestyle{acmnumeric}     %% For numeric citations
%\setcitestyle{nosort}      %% With 'acmnumeric', to disable automatic
                            %% sorting of references within a single citation;
                            %% e.g., \cite{Smith99,Carpenter05,Baker12}
                            %% rendered as [14,5,2] rather than [2,5,14].
%\setcitesyle{nocompress}   %% With 'acmnumeric', to disable automatic
                            %% compression of sequential references within a
                            %% single citation;
                            %% e.g., \cite{Baker12,Baker14,Baker16}
                            %% rendered as [2,3,4] rather than [2-4].


%%%%%%%%%%%%%%%%%%%%%%%%%%%%%%%%%%%%%%%%%%%%%%%%%%%%%%%%%%%%%%%%%%%%%%
%% Note: Authors migrating a paper from traditional SIGPLAN
%% proceedings format to PACMPL format must update the
%% '\documentclass' and topmatter commands above; see
%% 'acmart-pacmpl-template.tex'.
%%%%%%%%%%%%%%%%%%%%%%%%%%%%%%%%%%%%%%%%%%%%%%%%%%%%%%%%%%%%%%%%%%%%%%


%% Some recommended packages.
\usepackage{booktabs}   %% For formal tables:
                        %% http://ctan.org/pkg/booktabs
\usepackage{subcaption} %% For complex figures with subfigures/subcaptions
                        %% http://ctan.org/pkg/subcaption


\usepackage{amsmath}
\usepackage{amsthm}
\usepackage{amssymb}
\usepackage{mathtools}

% Theorem styles 

\theoremstyle{plain}
\newtheorem{thm}{Theorem}

\theoremstyle{definition}
\newtheorem{prop}[thm]{Proposition}

\theoremstyle{remark}
\newtheorem{claim}[thm]{Claim}

\theoremstyle{remark}
\newtheorem{corollary}[thm]{Corollary}

\theoremstyle{remark}
\newtheorem{rem}[thm]{Remark}

% Definitions on their own counter

\theoremstyle{definition}
\newtheorem{defn}{Definition}

% Notation

% General
\newcommand{\defeq}{\coloneqq}
\newcommand{\cat}[1]{\mathbf{#1}}
\newcommand{\equalizer}[2]{Eq(#1, #2)}
\newcommand{\powerset}[1]{\mathcal{P}(#1)}

% Change structures
\newcommand{\cplus}{\oplus}
\newcommand{\cpluss}{\boxplus}
\newcommand{\cplusss}{\odot}
\newcommand{\cminus}{\ominus}

\newcommand{\cstruct}[3]{(#1,#2,#3)}
\newcommand{\changes}[1]{\Delta #1}
\newcommand{\change}[1]{\delta #1}

\newcommand{\discrete}{\emptyset}

\newcommand{\derive}[1]{#1'}
\newcommand{\supderive}[1]{#1_\uparrow}
\newcommand{\supderiveM}[1]{#1_{\uparrow\uparrow}}
\newcommand{\subderive}[1]{#1_\downarrow}
\newcommand{\subderiveM}[1]{#1_{\downarrow\downarrow}}

\newcommand{\difffunc}{\rightarrow_{D}}

% Algebra
\newcommand{\superpose}{\circledast}

% Orders
\newcommand{\reachOrder}{\leq_R}
\newcommand{\changeOrder}{\leq_\Delta}
\newcommand{\fineOrder}{\leq_D}
\newcommand{\minusOrder}{\leq_\cminus}

\begin{document}

\author{
  Alex Eyers-Taylor\\
  Semmle Ltd.
  \and
  Michael Peyton Jones\\
  Semmle Ltd.
  \and
  Mario Alvarez Picallo\\
  University of Oxford
}

%%% Author information
%% Contents and number of authors suppressed with 'anonymous'.
%% Each author should be introduced by \author, followed by
%% \authornote (optional), \orcid (optional), \affiliation, and
%% \email.
%% An author may have multiple affiliations and/or emails; repeat the
%% appropriate command.
%% Many elements are not rendered, but should be provided for metadata
%% extraction tools.

\author{Alex Eyers-Taylor}
\affiliation{
  \position{Research Engineer}
  \institution{Semmle Ltd}            %% \institution is required
  \country{United Kingdom}                    %% \country is recommended
}
\email{alex@semmle.com}          %% \email is recommended

\author{Michael Peyton Jones}
\affiliation{
  \position{Research Engineer}
  \institution{Semmle Ltd}            %% \institution is required
  \country{United Kingdom}                    %% \country is recommended
}
\email{michael@semmle.com}          %% \email is recommended

\author{Mario Alvarez Picallo}
\affiliation{
  \position{PhD Student}
  \department{Computer Science}
  \institution{University of Oxford}            %% \institution is required
  \country{United Kingdom}                    %% \country is recommended
}
\email{mario.alvarez-picallo@cs.ox.ac.uk}          %% \email is recommended


%% 2012 ACM Computing Classification System (CSS) concepts
%% Generate at 'http://dl.acm.org/ccs/ccs.cfm'.
\begin{CCSXML}
<ccs2012>
<concept>
<concept_id>10011007.10011006.10011008</concept_id>
<concept_desc>Software and its engineering~General programming languages</concept_desc>
<concept_significance>500</concept_significance>
</concept>
<concept>
<concept_id>10003456.10003457.10003521.10003525</concept_id>
<concept_desc>Social and professional topics~History of programming languages</concept_desc>
<concept_significance>300</concept_significance>
</concept>
</ccs2012>
\end{CCSXML}

\ccsdesc[500]{Software and its engineering~General programming languages}
\ccsdesc[300]{Social and professional topics~History of programming languages}
%% End of generated code

%% Keywords
%% comma separated list
\keywords{incremental computation, performance, Datalog}  %% \keywords are mandatory in final camera-ready submission

\title{Something about change structures}

\maketitle

\section{Introduction}

The paper of Cai et al ~\cite{cai2014changes} defines compelling notions
of change structure and derivative, and use them to provide a composable framework for
evaluating incremental lambda calculus programs provided you have a ``plugin'' for the
primitive operations of your program.

This provides an excellent foundation for using change structures, but there are
many questions remaining about change structures themselves. What are their
algebraic properties? Can we define change structures on more varied kinds of
structures? What are the interactions with orders on the base set?

There are two major motivational targets for this exercise. The first is
lattices. Incremental computation is especially attractive for database or logic
programming languages (Datalog, SQL, etc.), where it can produce drastically
smaller results than naive evaluation. We therefore want to understand
incremental computation in its most general form on lattices, so that we have
maximum flexibility when bringing it to bear on a particular application. For
applications to Datalog in particular, see (cite forthcoming paper, or perhaps
my arxiv preprint).

The calculus of differential geometry is significantly more abstract than real
calculus. The connection to change structures is still tenuous, but there are
telling similarities, and it is our eventual goal to find a sufficiently general
notion of derivative that will encompass differential geometry as well.

The major contributions of this paper are:
\begin{itemize}
  \item A generalised definition of change structures and derivatives
    (\ref{sec:change-structures}).
  \item A description of the algebraic properties of change structures (\ref{sec:algebra}), including
    a definition of function changes that allows them to form a Cartesian closed
    category.
  \item A description of the properties of change structures over posets
    (\ref{sec:order}), including theorems that characterize the range of
    possible derivatives available, and conditions under which compound change
    structures may admit derivatives.
\end{itemize}

\section{Change structures}
\label{sec:change-structures}

\begin{defn}[Change structures]

  A \textit{change structure} is defined as:

  $$\mathcal{A} \defeq \cstruct{A}{\changes{A}}{\cplus}$$

  where $A$ is an object in some category $\cat{C}$, $\changes{A}$ is a semigroup, and $\cplus$ gives a semigroup action on $A$.

  We will call $A$ the base set (although it may not be a set in general), and $\changes{A}$ the change set of the change structure.
\end{defn}

Elements in the change set represent changes that can be made to elements in the
base set, with the semigroup action being the operation that ``applies'' the
change. The requirement that the change set be a semigroup is convenient but in
fact inessential: given any set with an action on the base set, we can take the
free semigroup over the action set to obtain a semigroup action (in fact, we can
easily extend to a monoid as well).

The fact that the change set is a semigroup action reveals the heart of the
change structure definition: the change set is a representation of a
\textit{transformation semigroup} over the base set, with the semigroup
operation simply being composition.

Here are some recurring examples of changes structures:
\begin{itemize}
  \item $A_\discrete \defeq \cstruct{A}{\emptyset}{\emptyset}$, the discrete change structure on any base set
  \item $\mathbb{Z}_1 \defeq \cstruct{\mathbb{Z}}{\mathbb{N}}{+}$
  \item $\mathbb{Z}_2 \defeq \cstruct{\mathbb{Z}}{\mathbb{N}}{-}$
  \item $\mathbb{Z}_3 \defeq \cstruct{\mathbb{Z}}{\mathbb{Z}}{+}$
  \item $F_2 \defeq$ integers modulo 2 with plus
\end{itemize}

Given change structures and functions between them, we have a natural notion of a derivative, following ~\cite{cai2014changes}:

\begin{defn}[Derivatives]
  A \textit{derivative} of a function $f: A_\cplus \rightarrow B_\cpluss$ is a function $\derive{f}: A \times \changes{A} \rightarrow
  \changes{B}$ such that
  $$f(a \cplus \change{a}) = f(a) \cpluss \derive{f}(a, \change{a})$$

  A function which has a derivative is called \textit{differentiable}.
\end{defn}

Derivatives need not be unique, in general, so we will speak of ``a''
derivative. 

\begin{thm}[The Chain Rule]
  Let $f: A_\cplus \rightarrow B_\cpluss$, $g: B_\cpluss \rightarrow C_\cplusss$ be differentiable functions. Then $g \circ f$ is also
  differentiable, with derivative given by
   $$\derive{(g \circ f)}(x, \change{x}) = \derive{g}\left(f(x), \derive{f}(x, \change{x})\right)$$
\end{thm}
\begin{proof}
  By equivalence:
  \begin{itemize}
    \item[ ]$(g \circ f)(x) \cplusss \derive{g}\left(f(x), \derive{f}(x,\change{x})\right)$
    \item[=]$(g(f(x)) \cplusss \derive{g}\left(f(x), \derive{f}(x,\change{x})\right)$
    \item[=]$g\left(f(x) \cpluss \derive{f}(x, \change{x}) \right)$
    \item[=]$g\left(f(x \cplus \change{x})\right)$
    \item[=]$(g \circ f)(x \cplus \change{x})$
  \end{itemize}
  Therefore $\derive{g}\left(f(x), \derive{f}(x, \change{x})\right)$ is a
  derivative for $(g \circ f)$.
\end{proof}

Why call this the ``chain rule''? It doesn't have quite the same structure as
the chain rule in real calculus ($\derive{(g \circ f)}(x) = (\derive{g} \circ f)
(x) \cdot \derive{f}(x)$), but it does have the same structure as the chain rule
from differential geometry ($\derive{(g \circ f)}(x, \textbf{v}) = \derive{g}
(f(x), \derive{f}(x, \textbf{v}))$), which will turn out to be a recurring connection.

\subsection{Minus operators and completeness}

Cai et al ~\cite{cai2014changes} include a ``minus operator'' in their definition of change structures. 

\begin{defn}[Minus operator]
  A \textit{minus operator} is a function $\cminus: A \times A \rightarrow \changes{A}$ such that $a \cplus (b \cminus a) = b$.
\end{defn}

We have omitted minus operators from our definition because
there are many interesting change structures that do not have them (for example,
$\mathbb{Z}_3$ does, but $\mathbb{Z}_1$ does not).

\begin{defn}[Completeness]
  A change structure is \textit{complete} if for any $a, b \in A$, there is
  a change $\change{a} \in \changes{A}$ such that $a \cplus \change{a} = b$.
\end{defn}

\begin{prop}[Completeness equivalences]
  Let $A$ be a change structure. Then the following are equivalent:
  \begin{itemize}
    \item $A$ is complete.
    \item The semigroup action is transitive.
    \item There is a minus operator on $A$.
  \end{itemize}
\end{prop}

\begin{defn}[Minus derivative]
  Given a minus operator $\cminus$, we have a derivative for any function $f$,
  defined as
  $$\derive{f}_\cminus(a, \change{a}) \defeq f(a \cplus \change{a}) \cminus f(a)$$
\end{defn}

\subsection{Extensionality}

\begin{defn}[Extensionality]
  A change structure is \textit{extensional} if $\forall a \in A. a \cplus \change{a}
  = a \cplus \change{b}$ implies that $\change{a} = \change{b}$.
\end{defn}

Many change structures are not extensional, for example $F_2$.

\begin{prop}[Extensionality equivalences]
  Let $A$ be a change structure. Then the following are equivalent.
  \begin{itemize}
    \item $A$ is extensional.
    \item The semigroup action is faithful.
    \item The transformation semigroup obtained from the semigroup action is
      isomorphic to it.
  \end{itemize}
\end{prop}

Extensionality gives us uniqueness of derivatives.

\begin{prop}
  Let $B$ be an extensional change structure, and $f: A \rightarrow B$. Then $f$ has at
  most one derivative.

  Conversely, if $\textrm{id}: B \rightarrow B$ has at most one derivative, then
  $B$ is extensional.
\end{prop}

\section{Algebra of change structures}
\label{sec:algebra}

\begin{defn}[Category of change structures]
  We define the category $\cat{CStruct}$ of change structures over objects in
  some category $\cat{C}$. The objects are
  change structures and the morphisms are differentiable functions. We denote
  the set of differentiable functions between $A$ and $B$ as $A \difffunc B$.
  
  We will usually leave the category $\cat{C}$ implicit, or refer to it as the
  underlying category.
\end{defn}

We can now see the difference between $\cat{CStruct}$ and the category of
S-acts $\cat{SAct}$: the objects of $\cat{SAct}$ all maintain the same monoid
structure, whereas we are interested in changing the structure of the act.

However, if we compare the definition of a $\cat{SAct}$ ``act-preserving''
homeomorphism (as in ~\cite{kilp2000monoids}) we can see that the structure is quite similar to our definition
of differentiability:

$$f(a \splus s) = f(a) \splus s$$

as opposed to

$$f(a \cplus s) = f(a) \cplus \derive{f}(a, s)$$

That is, we use $\derive{f}$ to transform the action element into the new
structure, whereas in $\cat{SAct}$ it simply remains the same.

In fact, $\cat{SAct}$ is a subcategory of $\cat{CStruct}$, where we only
consider change structures with change set $S$, and the only functions are those
whose derivative is $\lambda a. \lambda d. d$.

\begin{prop}[Products]
  Let $A = \cstruct{A}{\changes{A}}{\cplus}$ and $B =
  \cstruct{B}{\changes{B}}{\cpluss}$ be change structures, and suppose that the
  underlying category has products.

  Then $A \times B \defeq \cstruct{A \times B}{\changes{A} \times
  \changes{B}}{\cplus \times \cpluss}$ is their categorical product.
\end{prop}
\begin{proof}
  Let $Y$ be a change structure, and $f_1: Y \rightarrow A$, $f_2: Y
  \rightarrow B$ be morphisms.

  Then the product morphism in the underlying category, $f_1 \times f_2$ is the product
  morphism in $\cat{CStruct}$.

  First, we show it is a morphism (i.e.) is differentiable. It can easily be
  shown that $\derive{f_1} \times \derive{f_2}$ is a derivative of $f_1 \times f_2$.

  Commutativity and uniqueness follow from the corresponding properties of the
  product in the underlying category.
\end{proof}

\begin{prop}[Equalizers]
  Let $A$ and $B$ be change structures, $f, g: A \rightarrow B$ be morphisms
  between them, and suppose that the underlying category has equalizers.

  Let $\equalizer{f}{g}$ be the equalizer of $f$ and $g$ in the underlying category.

  Then $\equalizer{f}{g}_\discrete$ is an equalizer for $f$ and $g$
  in $\cat{CStruct}$.
\end{prop}
\begin{proof}
  The equalizer morphism is differentiable, since any function from a discrete change
  structure is differentiable (see below), so it is a valid morphism in $\cat{CStruct}$.

  Equalization and uniqueness follow from the corresponding properties of the
  equalizer in the underlying category.
\end{proof}

\begin{thm}
  $\cat{CStruct}$ has all finite limits if the underlying category does.
\end{thm}

\begin{prop}[Exponentials]
  Let $A_\cplus$ and $B_\cpluss$ be change structures, and suppose that the
  underlying category has exponentials.

  Then $\cstruct{A \difffunc B}{A
    \rightarrow \changes{B}}{\lambda d. \lambda f. \lambda a. f(a) \cpluss
    d(a)}$ is a change structure on $A \difffunc B$ and the exponential object.

  The semigroup structure on $A \rightarrow \changes{B}$ is the semigroup
  structure on $\changes{B}$ lifted pointwise, so we will typically reuse the
  change operator for $B$ for $A \rightarrow \changes{B}$.
\end{prop}
\begin{proof}
  We need to show that the evaluation map $ev: (A \difffunc B) \times A
  \rightarrow B$ is differentiable, the other properties follow from the
  properties of the exponential object in $\cat{Set}$.

  TODO: I actually have no idea if this is even right. We definitely need to
  talk about function changes at some point, regardless.
\end{proof}

\begin{thm}
  $\cat{CStruct}$ is a Cartesian closed category if the underlying category is.
\end{thm}

\subsection{Ordering change structures}

We can put an order on the change structures for a given base set as follows:

\begin{defn}[Change structure ordering]
  $A_\cplus \fineOrder A_\cpluss$ iff $\textrm{id}: A_\cplus \rightarrow A_\cpluss$ is differentiable.
\end{defn}

Transitivity of the order follows from the chain rule, and reflexivity is trivial.

This ordering is useful because it gives us a natural sense of the ``fineness''
of a change structure, much like the corresponding version in topology.

\begin{prop}
  If $f: A_\cplus \rightarrow B_\cpluss$ is differentiable, then
  \begin{itemize}
    \item if $A_\cplusss \fineOrder A_\cplus$ then $f: A_\cplusss \rightarrow
      B_\cpluss$ is differentiable.
    \item if $B_\cplus \fineOrder B_\cplusss$ then $f: A_\cplus \rightarrow
      B_\cplusss$ is differentiable.
  \end{itemize}
\end{prop}

That is, functions remain differentiable if the source change structure becomes
coarser, or the target change structure becomes finer (again, mirroring topology).

Furthermore, $\fineOrder$ also gives a fineness ordering on the reachability orders.

\begin{prop}
  If $a \reachOrder b$ in $A_\cplus$ and $A_\cplus \fineOrder A_\cpluss$, then $a \reachOrder b$ in $A_\cpluss$.
\end{prop}

\subsection{Superpositions}

As well as combining change structures entire, we can combine two different
change structures on the same underlying set.

\begin{defn}[Superposition]
  Let $A_\cplus = \cstruct{A}{\changes{A}_\cplus}{\cplus}$ and $A_\cpluss =
  \cstruct{A}{\changes{A}_\cpluss}{\cpluss}$ be change structures.

  Then the \textit{superposition} of $A_\cplus$ and $A_\cpluss$ is defined as:
  $$A_\cplus \superpose A_\cpluss \defeq \cstruct{A}{
    (\changes{A}_\cplus \times \changes{A}_\cpluss)^\ast}{\star^\ast}$$

  where $X^\ast$ is the set of finite sequences of $X$, and $a \star (d, e)
  \defeq a \cplus d \cpluss e$.
\end{defn}

Here we have used the ``trick'' mentioned earlier to ensure that our change
set has a semigroup structure: $\cstruct{A}{(\changes{A}_\cplus \times
  \changes{A}_\cpluss)}{\star}$ does not have a semigroup structure, so we take
the free extension to finite sequences.

In some cases we may be able to find a more compact representation of the
superposition change structure, for example $\mathbb{Z}_1 \superpose \mathbb{Z}_2$ is isomorphic to $\mathbb{Z}_3$.

Superposition is a useful construction, because it is a (weak) coproduct.

\begin{prop}
  $A \superpose B$ is a weak coproduct of $A$ and $B$.
\end{prop}
\begin{proof}
  TODO - also, I think we may need the component change structures to actually
  be monoidal here, so that we can lift $\change{b}$ to $(0, \change{b})$ in
  the superposition.
\end{proof}

However, the weakness is not terribly important when considering $\fineOrder$
(in particular, if we drop to the category corresponding to that order, the
coproduct is strong again, since there is at most one morphism between any two
objects). 

\begin{corollary}
  $A_\cplus \superpose A_\cpluss$ is the least upper bound of $A_\cplus$ and $A_\cpluss$ with respect to $\fineOrder$.
\end{corollary}

That is, the superposition is the weakest change structure that is finer than both
$A_\cplus$ and $A_\cpluss$.

This gives us a join-semilattice for change structures on a given set.

\begin{thm}[Change structure semilattice]
  Change structures on a base set $A$ form a bounded join-semilattice 
  ordered by $\fineOrder$, with the least element given by
  $A_\discrete$, and the join operation given by $\superpose$.
\end{thm}

All functions into a complete change structure are differentiable, so any
complete change structures on $A$ will be maximal elements of the lattice, while
the discrete change structure provides a minimal element.

\section{Order and change structures}
\label{sec:order}

\subsection{Orders on change sets}

There is a natural preorder on the base set of a change structure, given by reachability under the action.

\begin{defn}[Reachability order]
  $a \reachOrder b$ iff there is a $\change{a} \in \changes{A}$ such that $a \cplus
  \change{a} = b$.
\end{defn}

A complete change structure has a complete (TODO: is this the right word?) reachability order;
a discrete change structure has a discrete reachability order. If the change set
semigroup is a monoid, then the order is a full partial order.

\begin{prop}
  A function is differentiable iff it is monotonic with respect to the
  reachability order.
\end{prop}

\begin{corollary}
  Any function from a discrete change structure or into a complete change
  structure is differentiable.
\end{corollary}

\subsection{Orders on base sets}

A common structure which we want to compute changes on is a poset. For this
section we shall assume that all of our base sets are posets.

Firstly, we can define ``approximations'' to derivatives from both sides.

\begin{defn}
  Let $f: A_\cplus \rightarrow B_\cpluss$ be a function. Then a \textit{sup-derivative}
  of $f$ is a function $\supderive{f}$ such that
  $$f(a \cplus \change{a}) \leq f(a) \cpluss \supderive{f}(a, \change{a})$$
  
  Similarly, a \textit{sub-derivative} of $f$ is a function $\subderive{f}$ such that 
  $$f(a \cplus \change{a}) \geq f(a) \cpluss \subderive{f}(a, \change{a})$$

  A function with a sup-derivative is sup-differentiable, and a function with a
  sub-derivative is sub-differentiable.
\end{defn}

Some change structures always have sub- or sup-derivatives: for example $\mathbb{Z}_1$
always has sup-derivatives, and $\mathbb{Z}_2$ always has sub-derivatives.

\begin{prop}
  If $f$ is both a sub- and sup-derivative, then it is a derivative.
\end{prop}

Note that this is not the same as saying that if $f$ is both sub- and
sup-differentiable, then it is differentiable. The functions which provide the
sub- and sup-derivatives must coincide for that to be the case.

Secondly, if the base set of a change structure is a poset, then this gives us a natural
order on the change set.

\begin{defn}[Change order]
  $\change{a} \changeOrder \change{b}$ iff for all $a \in A$ it is the case that $a \cplus \change{a} \leq a \cplus \change{b}$.
\end{defn}

Alternatively, the change order is the largest (TODO: I think this is right?) order such that $\cplus$ is monotonic with
respect to its second argument.

If the change structure is extensional, then the order is antisymmetric, and a
full partial order.

Having a monotonic order on the changes is very useful.

\begin{thm}
  Let $f: A \rightarrow B$ be a function, and let $\changeOrder$ be a preorder on $\changes{B}$ such that $\cplus$ is monotonic with
  respect to it. Then let $\supderive{f}$ be a sub-derivative for $f$, and $h: A \times
  \changes{A} \rightarrow \changes{B}$ be a function such that
  $$\supderive{f} \changeOrder h$$
  Then $h$ is also a sup-derivative for $f$.

  Similarly, if $\subderive{f}$ is a sup-derivative for $f$ such that 
  $$h \changeOrder \subderive{f}$$
  Then $h$ is also a sub-derivative for $f$.
\end{thm}
\begin{proof}
  We prove the first case:
  \begin{itemize}
    \item[ ]$\supderive{f}(a, \change{a}) \changeOrder h(a, \change{a})$
    \item[$\Rightarrow$]\{ by monotonicity \}\\
      $f(a) \cplus \supderive{f}(a, \change{a}) \leq f(a) \cplus h(a, \change{a})$
    \item[$\Rightarrow$]\{ sup-derivative property \}\\
      $f(a \cplus \change{a}) \leq f(a) \cplus h(a, \change{a})$
  \end{itemize}

  The proof for the other case is symmetric.
\end{proof}

\begin{thm}[Sandwich lemma]
  \label{thm:sandwich}
  Let $\supderive{f}$ be a sup-derivative for $f$, $\subderive{f}$ be a sub-derivative for $f$, $\changeOrder$ be a preorder on $\changes{B}$ such that $\cplus$ is monotonic with
  respect to it, and $g$ be such that

  $$\supderive{f} \changeOrder g \changeOrder \subderive{f}$$

  Then $g$ is a derivative for $f$.
\end{thm}

In particular, this applies if $g$ and $h$ are themselves derivatives. Moreover,
although the condition of the theorem only requires the bounds to be sub- and
sup-derivatives, the conclusion of the theorem also applies to the bounds, so
they will always be full derivatives as well.

\subsection{Ascending and descending change structures}

Sub- and sup-derivatives alone are not quite enough to allow us to make
functions fully differentiable. We need some additional power.

\begin{defn}[Ascending and descending change structures]
  A change structure $A$ is \textit{ascending} if $a \leq b$ implies $a
  \reachOrder b$.

  A change structure $A$ is \textit{descending} if $a \leq b$ implies $b
  \reachOrder a$.
\end{defn}

Intuitively, an ascending change structure is one where you can produce
arbitrary changes that follow the partial order.

\begin{corollary}
  A change structure which is both ascending, descending, and has a minimal or
  maximal element is complete.
\end{corollary}

\begin{thm}
  Let $A$, $B$ be a change structures, $f: A \rightarrow B$ be a function. Then
  any of the following are sufficient for $f$ to be differentiable.
  \begin{itemize}
    \item $B$ is ascending, and $f$ is sub-differentiable.
    \item $B$ is descending, and $f$ is sup-differentiable.
    \item $B$ is ascending, descending, and has a minimal or maximal element.
  \end{itemize}
\end{thm}

In particular, we can construct change structures where functions are
differentiable by superposing multiple change structures that have some of the
properties that we want.

For example, $\mathbb{Z}_1$ is ascending (and has sup-derivatives), and
$\mathbb{Z}_2$ is descending (and has sub-derivatives), so $\mathbb{Z}_1
\superpose \mathbb{Z}_2 = \mathbb{Z}_3$ has derivatives.

\subsection{Maximal and minimal derivatives}

TODO: this section isn't very elegant

If we have a minus operator, then our change structure is complete and all
functions are differentiable. However, there may still be multiple derivatives
for a given function, and we can distinguish them using our order on the change
set.

\begin{defn}[Minus ordering]
  $\cminus_1 \minusOrder \cminus_2$ iff for all $a,b \in A$, $a \cminus_1 b
  \changeOrder a \cminus_2 b$.
\end{defn}

This orders our minus operators according to the size of the changes they
produce. 

\begin{prop}
  If $\cminus_1 \minusOrder \cminus_2$ then
  $\derive{f}_{\cminus_1} \changeOrder \derive{f}_{\cminus_2}$.
\end{prop}

\begin{prop}
  If $\cminus$ is a minimal (maximal) minus operator, then $\derive{f}_\cminus$
  is a minimal (maximal) derivative.
\end{prop}

This then gives us a full characterisation of the derivatives on a complete
change structure.

\begin{thm}[Characterization of derivatives]
\label{thm:derivativeCharacterization}
  Let $A$ be a change structure and $B$ be a complete change structure, let
  $f: A \rightarrow B$ be a function, and let $\subderiveM{f}$ and
  $\supderiveM{f}$ be minimal and maximal derivatives of $f$, respectively.
  Then the derivatives of $f$ are precisely
  the functions $\derive{f}$ such that
  $$\subderiveM{f} \changeOrder \derive{f} \changeOrder \supderiveM{f}$$
\end{thm}
\begin{proof}
  Follows easily from \ref{thm:sandwich} and minimality/maximality.
\end{proof}

This theorem gives us leeway when trying to pick a derivative: we can pick out the
bounds, and that tells us how much ``wiggle room'' we have. This is helpful
because some of the intermediary functions may be much easier to compute than
others, or convenient for other reasons.

\section{Lattices}

\begin{defn}
  Let $L$ be a join-semilattice. Then $L_\vee \defeq \cstruct{L}{L}{\vee}$ is a change
  structure on $L$.

  Similarly, if $L$ is a meet-semilattice, then $L_\wedge \defeq \cstruct{L}{L}{\wedge}$ is a change
  structure on $L$.
\end{defn}

\begin{prop}
  All of the following hold
  \begin{itemize}
    \item $L_\vee$ is ascending and has sup-derivatives.
    \item $L_\wedge$ is descending and has sub-derivatives.
    \item $L_\vee \superpose L_\wedge$ has derivatives.
  \end{itemize}
\end{prop}

As usual, the superposition change structure is a pain to work with, since it
consists of sequences of ``upwards'' and ``downwards'' changes. However, it does
give us a complete change structure for any lattice.

\subsection{Boolean algebras}

Boolean algebras give us a much more compact representation for the
superposition of $L_\vee$ and $L_\wedge$.

\begin{prop}
  Let $L$ be a Boolean algebra. Define
  $$L_\superpose \defeq \cstruct{L}{L \times L}{\twist}$$
  where
  $$a \twist (p, q) \defeq (a \vee p) \wedge \neg q$$
  and the semigroup action is
  $$(p, q) \splus (r, s) \defeq ((p \wedge \neg s) \vee r, (q \vee s) \wedge \neg r)$$

  Then $L_\superpose$ is isomorphic to $L_\vee \superpose L_\wedge$.
\end{prop}

We can think of $L_\superpose$ as tracking changes as pairs of ``upwards'' and
``downwards'' changes, where the semigroup action simply applies both.

Boolean algebras also have clean definitions for maximal and minimal minus
operators.

\begin{prop}
  Let $L$ be a Boolean algebra. Then
  $$a \cminus_\bot b = (a \wedge \neg b, b)$$
  $$a \cminus_\top b = (a, b \wedge \neg a)$$

  define minimal and maximal minus operators.
\end{prop}

In particular, \ref{thm:derivativeCharacterization} gives us clear bounds for
all the derivatives on Boolean algebras:

\begin{corollary}
  Let $L$ be a Boolean algebra with the $L_\superpose$ change structure, $A$ be
  a change structure, and $f: A \rightarrow
  L$ a function. Then the derivatives of $f$ are precisely those functions
  $\derive{f}$ such that
  $$
  (
    f(a \cplus \change{a}) \wedge \neg f(a), 
    f(a)
  )
  \changeOrder
  \derive{f}(a, \change{a})
  \changeOrder
  (
    f(a \cplus \change{a}), 
    f(a) \wedge \neg f(a \cplus \change{a})
  )
  $$
\end{corollary}

\section{Fixpoints}

\subsection{Incremental computation of fixpoints}

Derivatives give us a technique for computing fixpoints incrementally. Kleene's
fixpoint theorem tells us that fixpoints exist for monotone functions on dcpos, and also gives us
a simple procedure for computing them: start from $\bot$ and apply the function
until there is no change. However, this can be woefully inefficient.

In the Datalog literature, the approach of computing the fixpoint by bottom-up
iteration is called ``naive evaluation''. Naive evaluation has the property that
if it derives a fact at some iteration, it will derive that fact at each
subsequent iteration as well. This is obviously wasteful, and can turn what
should be a linear computation into a quadratic one.

The canonical solution to this problem is ``semi-naive evaluation'', which
attempts to derive only the new facts at each iteration. However, ``semi-naive''
as traditionally presented has some warts, and
the following theorem provides a generalization of it to any differentiable function over a
change structure. We will see the details of how to differentiate Datalog
semantics in section SOMETHING.

\begin{thm}[Incremental computation of iterated function applications]
\label{thm:diffIter}
  Let $L$ be a dcpo with least element $\bot$ and a change structure, $f: L \rightarrow L$ be
  differentiable. Define $F_i$ as follows:

  \begin{eqnarray*}
  F_0 & = & x\\
  \Delta F_0 & = & (\bot, \top) \\
  F_1 & = & f(F_0)\\
  \Delta F_1 & = & (F_1, F_1) \footnotemark \\
  F_{i+2} & = & F_{i+1} \cplus \Delta F_{i+2} \\
  \Delta F_{i+2} & = & \derive{f}(F_i, \Delta F_{i+1}) \\
  \end{eqnarray*}

  Then 
  $$f^i(x) = F_i$$
\end{thm}

\footnotetext{A number of different values will do here, in general one can use $F_1
\cminus F_0$ given a valid minus operator.}

\begin{proof}
We proceed inductively, proving that $F_{i+2} = f(F_{i+1})$

\begin{itemize}
\item[ ]$F_{i+2}$
\item[=]
$
F_{i+1} \cplus \Delta F_{i+2}
$
\item[=]
$
F_{i+1} \cplus \derive{f}( F_i, \Delta F_{i+1})
$
\item[=] \{ by induction \}\\
$
f(F_i) \cplus \derive{f}(F_i, \Delta F_{i+1})
$
\item[=]
$
f(F_i \cplus \Delta F_{i+1})
$ 
\item[=]
$f(F_{i+1})$
\end{itemize}
\end{proof}

\begin{corollary}[Differential computation of fixed points]
\label{corollary:diffFP}
  Fixed points of differentiable functions which can be calculated by repeated
  function application can also be calculated incrementally.
\end{corollary}

\subsection{Derivatives of fixpoints}

The previous section has shown us how to use derivatives to compute fixpoints
more efficiently, but we might also want to take the derivative of a fixpoint
itself. The typical case for this will be where we have some fixed point
$$\mu X. F(E, X)$$
and we now wish to apply a change to $E$ and compute
$$\mu X. F(E \cplus \change{E}, X)$$

In Datalog this would allow us to update a recursively defined relation given an
update to a non-recursive dependency of it, such as the base database. In the
traditional databases literature, this amounts to a solution to the ``recursive
view update problem''.

However, this requires us to have a derivative for the fixpoint operator $\mu$.

\begin{thm}[Derivatives of fixpoints]
  Let $\fixpoint_X : (X \rightarrow X) \rightarrow X$ be a (differentiable?)
  fixpoint operator, $A$ be a change structure, $f: A \rightarrow A$ a differentiable function,
  and $\change{f} \in \changes{(A \rightarrow A)}$ be a change to $f$.

  Define
  $$
  \derive{\fixpoint_A}(f, \change{f}) \defeq
  \fixpoint_{\changes{A}}(
    \lambda w .
      \derive{f}(\fixpoint_A(f), w)
      \splus
      \change{f}(\fixpoint_A(f) \cplus w)
  )
  $$
  Then $\fixpoint_A(f) \cplus \derive{\fixpoint_A}(f, \change{f})$ is a fixpoint of $f \cplus \change{f}$.

  Furthermore, if $A$ is a partial order, $\fixpoint$ computes the least
  fixpoint, $\cplus$ is monotonic with respect to the order on $\changes{A}$,
  then the previous expression is also a least fixpoint and
  $\derive{\fixpoint_A}$ is a derivative.
\end{thm}

\newcommand{\thefixpoint}{F}
\newcommand{\theadjustment}{adjust}

\begin{proof}
  For the first part, let
  $\thefixpoint = \fixpoint_A(f)$ and
  $\theadjustment(F, f, \change{f}) = \lambda w . \derive{f}(F, w) \splus
  \change{f}(F \cplus w)$.
  Then
  \begin{itemize}
  \item[ ]
    $
    (f \cplus \change{f})(\thefixpoint \cplus \derive{\fixpoint_A}(f, \change{f}))
    $
  \item[=]\{ definition of $\cplus$ for $\changes{(A \rightarrow A)}$ \}\\
    $
    f(\thefixpoint \cplus \derive{\fixpoint_A}(f, \change{f}))
    \cplus
    \change{f}(\thefixpoint \cplus \derive{\fixpoint_A}(f, \change{f}))
    $
  \item[=]\{ derivative property of $\derive{f}$ \}\\
    $
    f(\thefixpoint)
    \cplus
    \derive{f}(\thefixpoint, \derive{\fixpoint_A}(f, \change{f}))
    \cplus
    \change{f}(\thefixpoint \cplus \derive{\fixpoint_A}(f, \change{f}))
    $
  \item[=]\{ semigroup action property \}\\
    $
    f(\thefixpoint)
    \cplus
    \derive{f}(\thefixpoint, \derive{\fixpoint_A}(f, \change{f}))
    \splus
    \change{f}(\thefixpoint \cplus \derive{\fixpoint_A}(f, \change{f}))
    $
  \item[=]\{ definition of $\theadjustment$, $\derive{\fixpoint_A}$ \}\\
    $
    f(\thefixpoint)
    \cplus
    \theadjustment(F, f, \change{f})(\fixpoint_{\changes{A}}(\theadjustment(F, f, \change{f})))
    $
  \item[=]\{ rolling both fixpoints \}\\
    $
    \thefixpoint
    \cplus
    \fixpoint_{\changes{A}}(\theadjustment(F, f, \change{f}))
    $
  \item[=]\{ definition of $\derive{\fixpoint_A}$ \}\\
    $
    \thefixpoint
    \cplus
    \derive{\fixpoint_A}(f, \change{f})
    $
  \end{itemize}

  Hence $\derive{\fixpoint_A}(f, \change{f})$ is a fixpoint of $f \cplus \change{f}$.

  For the second part, suppose that $F'$ is a fixpoint of $f \cplus \change{f}$
  Then ???
\end{proof}

Sadly, this is far from simple to compute. It requires a fixpoint over the
change structure partial order, which may not be possible (depending on our
change structure). However, for lattices the $L_\superpose$ change structure is
itself a lattice, so we can compute fixpoints over it.

It may not be too expensive to compute a fixpoint, since the alternative
``update strategy'' is the worst possible one: throw everything away and start
again (which will itself require a fixpoint computation).

\section{Related work}

The seminal paper in this area is ~\cite{cai2014changes}. We use the notions
defined in that excellent paper heavily, but we deviate in two regards: the use of
dependent types, and the nature of function changes.

These two issues are linked, because part of the reason that Cai et al need
dependently typed changes is in order to handle their notion of function
changes.

Our notion of function changes is different to Cai's because Cai requires the
function changes to behave like derivatives. To our eyes, this confuses two ways
in which functions can change: the derivative gives the change in the function
at a point, as the point moves; whereas a change in the function itself is a
global change to the function at all points.

Furthermore, our notion of function changes provides a more natural structure,
in that it allows us to make the category of change structures into a cartesian
closed category.

Our function changes do not need to be dependently typed, so we do not need a
dependently typed formulation for the cases we have been considering. However,
if we want to extend this formulation to differential geometry we will need
dependently typed changes, since the type of the tangent space is
dependent on the point at which the tangents are taken.

In fact, a slightly different dependently-typed formulation seems promising, since it would allow us
to draw out a 2-categorical interpretation of change structures. For example, if
we consider our base set as a category, with changes $\Delta_a^b$ between $a$
and $b$ as morphisms between $a$ and $b$, then a function is
differentiable iff it is a functor on that category (i.e. provides a change in $\Delta_{f(a)}^{f(b)}$).

While this is a promising direction, it wasn't necessary for the material in
this paper, and would have complicated the exposition significantly, so we opted
to leave it for future work.

S-acts and their categorical structure have received a fair amount of attention
over the years (~\cite{kilp2000monoids} is a good overview). However, our work
is largely distinct from this, since we are interested in acts over different
monoids/semigroups, which necessitates the presence of derivatives.

Griffin?

\bibliography{paper}

\end{document}
