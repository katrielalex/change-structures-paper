\documentclass[english]{article}

\usepackage{hyperref}

\bibliographystyle{plain}

%%% For double-blind review submission, w/o CCS and ACM Reference (max submission space)
\documentclass[sigplan,10pt,review,anonymous,natbib=false]{acmart}\settopmatter{printfolios=true,printccs=false,printacmref=false}
%% For double-blind review submission, w/ CCS and ACM Reference
%\documentclass[sigplan,10pt,review,anonymous]{acmart}\settopmatter{printfolios=true}
%% For single-blind review submission, w/o CCS and ACM Reference (max submission space)
%\documentclass[sigplan,10pt,review]{acmart}\settopmatter{printfolios=true,printccs=false,printacmref=false}
%% For single-blind review submission, w/ CCS and ACM Reference
%\documentclass[sigplan,10pt,review]{acmart}\settopmatter{printfolios=true}
%% For final camera-ready submission, w/ required CCS and ACM Reference
%\documentclass[sigplan,10pt]{acmart}\settopmatter{}

%% Conference information
%% Supplied to authors by publisher for camera-ready submission;
%% use defaults for review submission.
\acmConference[PL'17]{ACM SIGPLAN Conference on Programming Languages}{January 01--03, 2017}{New York, NY, USA}
\acmYear{2017}
\acmISBN{} % \acmISBN{978-x-xxxx-xxxx-x/YY/MM}
\acmDOI{} % \acmDOI{10.1145/nnnnnnn.nnnnnnn}
\startPage{1}

%% Copyright information
%% Supplied to authors (based on authors' rights management selection;
%% see authors.acm.org) by publisher for camera-ready submission;
%% use 'none' for review submission.
\setcopyright{none}
%\setcopyright{acmcopyright}
%\setcopyright{acmlicensed}
%\setcopyright{rightsretained}
%\copyrightyear{2017}           %% If different from \acmYear

%% Bibliography style
\bibliographystyle{ACM-Reference-Format}
%% Citation style
%\citestyle{acmauthoryear}  %% For author/year citations
%\citestyle{acmnumeric}     %% For numeric citations
%\setcitestyle{nosort}      %% With 'acmnumeric', to disable automatic
                            %% sorting of references within a single citation;
                            %% e.g., \cite{Smith99,Carpenter05,Baker12}
                            %% rendered as [14,5,2] rather than [2,5,14].
%\setcitesyle{nocompress}   %% With 'acmnumeric', to disable automatic
                            %% compression of sequential references within a
                            %% single citation;
                            %% e.g., \cite{Baker12,Baker14,Baker16}
                            %% rendered as [2,3,4] rather than [2-4].


%%%%%%%%%%%%%%%%%%%%%%%%%%%%%%%%%%%%%%%%%%%%%%%%%%%%%%%%%%%%%%%%%%%%%%
%% Note: Authors migrating a paper from traditional SIGPLAN
%% proceedings format to PACMPL format must update the
%% '\documentclass' and topmatter commands above; see
%% 'acmart-pacmpl-template.tex'.
%%%%%%%%%%%%%%%%%%%%%%%%%%%%%%%%%%%%%%%%%%%%%%%%%%%%%%%%%%%%%%%%%%%%%%


%% Some recommended packages.
\usepackage{booktabs}   %% For formal tables:
                        %% http://ctan.org/pkg/booktabs
\usepackage{subcaption} %% For complex figures with subfigures/subcaptions
                        %% http://ctan.org/pkg/subcaption


\usepackage{amsmath}
\usepackage{amsthm}
\usepackage{amssymb}
\usepackage{mathtools}

% Theorem styles 

\theoremstyle{plain}
\newtheorem{thm}{Theorem}

\theoremstyle{definition}
\newtheorem{prop}[thm]{Proposition}

\theoremstyle{remark}
\newtheorem{claim}[thm]{Claim}

\theoremstyle{remark}
\newtheorem{corollary}[thm]{Corollary}

\theoremstyle{remark}
\newtheorem{rem}[thm]{Remark}

% Definitions on their own counter

\theoremstyle{definition}
\newtheorem{defn}{Definition}

% Notation

% General
\newcommand{\defeq}{\coloneqq}
\newcommand{\cat}[1]{\mathbf{#1}}
\newcommand{\equalizer}[2]{Eq(#1, #2)}
\newcommand{\powerset}[1]{\mathcal{P}(#1)}

% Change structures
\newcommand{\cplus}{\oplus}
\newcommand{\cpluss}{\boxplus}
\newcommand{\cplusss}{\odot}
\newcommand{\cminus}{\ominus}

\newcommand{\cstruct}[3]{(#1,#2,#3)}
\newcommand{\changes}[1]{\Delta #1}
\newcommand{\change}[1]{\delta #1}

\newcommand{\discrete}{\emptyset}

\newcommand{\derive}[1]{#1'}
\newcommand{\supderive}[1]{#1_\uparrow}
\newcommand{\supderiveM}[1]{#1_{\uparrow\uparrow}}
\newcommand{\subderive}[1]{#1_\downarrow}
\newcommand{\subderiveM}[1]{#1_{\downarrow\downarrow}}

\newcommand{\difffunc}{\rightarrow_{D}}

% Algebra
\newcommand{\superpose}{\circledast}

% Orders
\newcommand{\reachOrder}{\leq_R}
\newcommand{\changeOrder}{\leq_\Delta}
\newcommand{\fineOrder}{\leq_D}
\newcommand{\minusOrder}{\leq_\cminus}

\begin{document}

\author{
  Alex Eyers-Taylor\\
  Semmle Ltd.
  \and
  Michael Peyton Jones\\
  Semmle Ltd.
  \and
  Mario Alvarez Picallo\\
  University of Oxford
}

%%% Author information
%% Contents and number of authors suppressed with 'anonymous'.
%% Each author should be introduced by \author, followed by
%% \authornote (optional), \orcid (optional), \affiliation, and
%% \email.
%% An author may have multiple affiliations and/or emails; repeat the
%% appropriate command.
%% Many elements are not rendered, but should be provided for metadata
%% extraction tools.

\author{Alex Eyers-Taylor}
\affiliation{
  \position{Research Engineer}
  \institution{Semmle Ltd}            %% \institution is required
  \country{United Kingdom}                    %% \country is recommended
}
\email{alex@semmle.com}          %% \email is recommended

\author{Michael Peyton Jones}
\affiliation{
  \position{Research Engineer}
  \institution{Semmle Ltd}            %% \institution is required
  \country{United Kingdom}                    %% \country is recommended
}
\email{michael@semmle.com}          %% \email is recommended

\author{Mario Alvarez Picallo}
\affiliation{
  \position{PhD Student}
  \department{Computer Science}
  \institution{University of Oxford}            %% \institution is required
  \country{United Kingdom}                    %% \country is recommended
}
\email{mario.alvarez-picallo@cs.ox.ac.uk}          %% \email is recommended


%% 2012 ACM Computing Classification System (CSS) concepts
%% Generate at 'http://dl.acm.org/ccs/ccs.cfm'.
\begin{CCSXML}
<ccs2012>
<concept>
<concept_id>10011007.10011006.10011008</concept_id>
<concept_desc>Software and its engineering~General programming languages</concept_desc>
<concept_significance>500</concept_significance>
</concept>
<concept>
<concept_id>10003456.10003457.10003521.10003525</concept_id>
<concept_desc>Social and professional topics~History of programming languages</concept_desc>
<concept_significance>300</concept_significance>
</concept>
</ccs2012>
\end{CCSXML}

\ccsdesc[500]{Software and its engineering~General programming languages}
\ccsdesc[300]{Social and professional topics~History of programming languages}
%% End of generated code

%% Keywords
%% comma separated list
\keywords{incremental computation, performance, Datalog}  %% \keywords are mandatory in final camera-ready submission

\title{Something about change structures}

\author{
  Alex Eyers-Taylor\\
  Semmle Ltd.
  \and
  Michael Peyton Jones\\
  Semmle Ltd.
  \and
  Mario Alvarez Picallo\\
  University of Oxford
}

\maketitle

\section{Introduction}

\section{Change structures}

TODO: decide if we're going to use ``change structure'' or ``forward algebra''
\begin{defn}[Change structures]
  A \textit{change structure} is defined as:

  $$\mathcal{A} \defeq \cstruct{A}{\changes{A}}{\cplus}$$

  where $A$ is an object in some category $\cat{C}$, $\changes{A}$ is a semigroup, and $\cplus$ gives a semigroup action on $A$.

  We will call $A$ the base set (although it may not be a set in general), and $\changes{A}$ the change set of the change structure.
\end{defn}

Elements in the change set represent changes that can be made to elements in the
base set, with the semigroup action being the operation that ``applies'' the
change. The requirement that the change set be a semigroup is convenient but in
fact inessential: given any set with an action on the base set, we can take the
free semigroup over the action set to obtain a semigroup action.

Some examples of changes structures are:
\begin{itemize}
  \item $A_\discrete \defeq \cstruct{A}{\emptyset}{\emptyset}$, the discrete
    change structure on any base set.
  \item $\mathbb{Z}_1 \defeq \cstruct{\mathbb{Z}}{\mathbb{N}}{+}$
  \item $\mathbb{Z}_2 \defeq \cstruct{\mathbb{Z}}{\mathbb{N}}{-}$
  \item $\mathbb{Z}_3 \defeq \cstruct{\mathbb{Z}}{\mathbb{Z}}{+}$
  \item $F_2 \defeq$ integers modulo 2 with plus. TODO typeset this nicely
\end{itemize}

Given change structures and functions between them, we have a natural notion of a derivative, following ~\cite{cai2014changes}:

\begin{defn}[Derivatives]
  A \textit{derivative} of a function $f: A_\cplus \rightarrow B_\cpluss$ is a function $\derive{f}: A \times \changes{A} \rightarrow
  \changes{B}$ such that
  $$f(a \cplus \change{a}) = f(a) \cpluss \derive{f}(a, \change{a})$$

  A function which has a derivative is called \textit{differentiable}.
\end{defn}

Derivatives need not be unique, in general, so we will speak of ``a'' derivative.

TODO: tightness

\begin{thm}[The Chain Rule]
  Let $f: A_\cplus \rightarrow B_\cpluss$, $g: B_\cpluss \rightarrow C_\cplusss$ be differentiable functions. Then $g \circ f$ is also
  differentiable, with derivative given by
   $$\derive{(g \circ f)}(x, \change{x}) = \derive{g}\left(f(x), \derive{f}(x, \change{x})\right)$$
\end{thm}
\begin{proof}
  By equivalence:
  \begin{itemize}
    \item[ ]$(g \circ f)(x) \cplusss \derive{g}\left(f(x), \derive{f}(x,\change{x})\right)$
    \item[=]$(g(f(x)) \cplusss \derive{g}\left(f(x), \derive{f}(x,\change{x})\right)$
    \item[=]$g\left(f(x) \cpluss \derive{f}(x, \change{x}) \right)$
    \item[=]$g\left(f(x \cplus \change{x})\right)$
    \item[=]$(g \circ f)(x \cplus \change{x})$
  \end{itemize}
  Therefore $\derive{g}\left(f(x), \derive{f}(x, \change{x})\right)$ is a
  derivative for $(g \circ f)$.
\end{proof}

Why call this the ``chain rule''? It doesn't have quite the same structure as
the chain rule in real calculus ($\derive{(g \circ f)}(x) = (\derive{g} \circ f)
(x) \cdot \derive{f}(x)$), but it does have the same structure as the chain rule
from differential geometry ($\derive{(g \circ f)}(x, \textbf{v}) = \derive{g}
(f(x), \derive{f}(x, \textbf{v}))$), which will turn out to be a recurring connection.

\subsection{Minus operators and completeness}

Cai et al include a ``minus operator'' in their definition of change structures. 

\begin{defn}[Minus operator]
  A \textit{minus operator} is a function $\cminus: A \times A \rightarrow \changes{A}$
  such that $a \cplus (b \cminus a) = b$.
\end{defn}

We have omitted minus operators from our definition because
there are many interesting change structures that do not have them (for example,
$\mathbb{Z}_3$ does, but $\mathbb{Z}_1$ does not).

\begin{defn}[Completeness]
  A change structure is \textit{complete} if for any $a, b \in A$, there is
  a change $\change{a} \in \changes{A}$ such that $a \cplus \change{a} = b$.
\end{defn}

\begin{prop}[Completeness equivalences]
  Let $A$ be a change structure. Then the following are equivalent:
  \begin{itemize}
    \item $A$ is complete.
    \item The semigroup action is transitive.
    \item There is a minus operator on $A$.
  \end{itemize}
\end{prop}

\subsection{Extensionality}

\begin{defn}[Extensionality]
  A change structure is \textit{extensional} if $\forall a \in A. a \cplus \change{a}
  = a \cplus \change{b}$ implies that $\change{a} = \change{b}$.
\end{defn}

Many change structures are not extensional, for example $F_2$.

\begin{prop}[Extensionality equivalences]
  Let $A$ be a change structure. Then the following are equivalent.
  \begin{itemize}
    \item $A$ is extensional.
    \item The semigroup action is faithful.
    \item The transformation semigroup obtained from the semigroup action is
      isomorphic to it.
  \end{itemize}
\end{prop}

\section{Algebra of change structures}

\begin{defn}[Category of change structures]
  We define the category $\cat{CStruct}$ of change structures over objects in
  some category $\cat{C}$. The objects are
  change structures and the morphisms are differentiable functions. We denote
  the set of differentiable functions between $A$ and $B$ as $A \difffunc B$.
  
  We will usually leave the category $\cat{C}$ implicit, or refer to it as the
  underlying category.
\end{defn}

\begin{prop}[Products]
  Let $\mathcal{A} = \cstruct{A}{\changes{A}}{\cplus}$ and $\mathcal{B} =
  \cstruct{B}{\changes{B}}{\cpluss}$ be change structures, and suppose that the
  underlying category has products.

  Then $\mathcal{A} \times \mathcal{B} \defeq \cstruct{A \times B}{\changes{A} \times
  \changes{B}}{\cplus \times \cpluss}$ is their categorical product.
\end{prop}
\begin{proof}
  Let $Y$ be a change structure, and $f_1: Y \rightarrow \mathcal{A}$, $f_2: Y
  \rightarrow \mathcal{B}$ be morphisms.

  Then the product morphism in the underlying category, $f_1 \times f_2$ is the product
  morphism in $\cat{CStruct}$.

  First, we show it is a morphism (i.e.) is differentiable. It can easily be
  shown that $\derive{f_1} \times \derive{f_2}$ is a derivative of $f_1 \times f_2$.

  Commutativity and uniqueness follow from the corresponding properties of the
  product in the underlying category.
\end{proof}

\begin{prop}[Equalizers]
  Let $A$ and $B$ be change structures, $f, g: A \rightarrow B$ be morphisms
  between them, and suppose that the underlying category has equalizers.

  Let $\equalizer{f}{g}$ be the equalizer of $f$ and $g$ in the underlying category.

  Then $\equalizer{f}{g}_\discrete$ is an equalizer for $f$ and $g$
  in $\cat{CStruct}$.
\end{prop}
\begin{proof}
  The equalizer morphism is differentiable, since any function from a discrete change
  structure is differentiable (see below), so it is a valid morphism in $\cat{CStruct}$.

  Equalization and uniqueness follow from the corresponding properties of the
  equalizer in the underlying category.
\end{proof}

\begin{thm}
  $\cat{CStruct}$ has all finite limits if the underlying category does.
\end{thm}

\begin{prop}[Exponentials]
  Let $A_\cplus$ and $B_\cpluss$ be change structures, and suppose that the
  underlying category has exponentials.

  Then $\cstruct{A \difffunc B}{A
    \rightarrow \changes{B}}{\lambda d. \lambda f. \lambda a. f(a) \cpluss
    d(a)}$ is a change structure on $A \difffunc B$ and the exponential object.

  The semigroup structure on $A \rightarrow \changes{B}$ is the semigroup
  structure on $\changes{B}$ lifted pointwise, so we will typically reuse the
  change operator for $B$ for $A \rightarrow \changes{B}$.
\end{prop}
\begin{proof}
  We need to show that the evaluation map $ev: (A \difffunc B) \times A
  \rightarrow B$ is differentiable, the other properties follow from the
  properties of the exponential object in $\cat{Set}$.

  $$\derive{ev}((f, a), (\change{f}, \change{a})) = \derive{(f \cpluss
    \change{f})}(a \cplus {\change{a}})$$

  TODO: I actually have no idea if this is even right. We definitely need to
  talk about function changes at some point, regardless.
\end{proof}

\subsection{Ordering change structures}

We can put an order on the change structures for a given base set as follows:

\begin{defn}[Change structure ordering]
  $A_\cplus \fineOrder A_\cpluss$ iff $\textrm{id}: A_\cplus \rightarrow A_\cpluss$ is differentiable.
\end{defn}

Transitivity of the order follows from the chain rule, and reflexivity is trivial.

This ordering is useful because it gives us a natural sense of the ``fineness''
of a change structure, in that

\begin{prop}
  If $f: A_\cplus \rightarrow B$ is differentiable, and $A_\cplus \fineOrder
  A_\cpluss$ then $f: A_\cpluss \rightarrow B$ is differentiable.
\end{prop}

\subsection{Superpositions}

As well as combining change structures entire, we can combine two different
change structures on the same underlying set.

\begin{defn}[Superposition]
  Let $A_\cplus = \cstruct{A}{\changes{A}_\cplus}{\cplus}$ and $A_\cpluss =
  \cstruct{A}{\changes{A}_\cpluss}{\cpluss}$ be change structures.

  Then the \textit{superposition} of $A_\cplus$ and $A_\cpluss$ is defined as:
  $$A_\cplus \superpose A_\cpluss \defeq \cstruct{A}{
    (\changes{A}_\cplus \times \changes{A}_\cpluss)^\ast}{\star}$$

  where $X^\ast$ is the set of finite sequences of $X$, and $a \star (d, e)
  \defeq (\cpluss \circ \cplus)^\ast$.
\end{defn}

In some cases we may be able to find a more compact representation of the
superposition change structure, for example:

\begin{rem}
  The superposition $\mathbb{Z}_1 \superpose \mathbb{Z}_2$ is isomorphic to $\mathbb{Z}_3$.
\end{rem}

The superposition is a useful construction, because it is a (weak) coproduct.

\begin{prop}
  $A \superpose B$ is a weak coproduct of $A$ and $B$.
\end{prop}
\begin{proof}
  TODO
\end{proof}

However, the weakness is not terribly important when considering $\fineOrder$
(in particular, if we drop to the category corresponding to that order, the
coproduct is strong again, since there is at most one morphism between any two
objects). 

\begin{corollary}
  $A_\cplus \superpose A_\cpluss$ is the least upper bound of $A_\cplus$ and $A_\cpluss$ with respect to $\fineOrder$.
\end{corollary}

That is, the superposition is the weakest change structure that is finer than both
$A_\cplus$ and $A_\cpluss$.

This gives us a join-semilattice for change structures on a given set.

\begin{thm}[Change structure semilattice]
  Change structures on a base set $A$ form a join-semilattice 
  ordered by $\fineOrder$, with the least element given by
  $A_\discrete$, and the join operation given by $\superpose$.
\end{thm}

All functions into a complete change structure are differentiable, so any
complete change structures on $A$ will be maximal elements of the latice.

\section{Order and change structures}

\subsection{Orders on change sets}

There is a natural preorder on the base set of a change structure, given by reachability under the action.

\begin{defn}[Reachability order]
  $a \reachOrder b$ iff there is a $\change{a} \in \changes{A}$ such that $a \cplus
  \change{a} = b$.
\end{defn}

A complete change structure has a complete (TODO: is this the right word?) reachability order;
a discrete change structure has a discrete reachability order. If the change set
semigroup is a monoid, then the order is a full partial order.

\begin{prop}
  A function is differentiable iff it is monotonic with respect to the
  reachability order.
\end{prop}

\begin{corollary}
  Any function from a discrete change structure or into a complete change
  structure is differentiable.
\end{corollary}

\subsection{Orders on base sets}

A common structure which we want to compute changes on is a poset. For this
section we shall assume that all of our base sets are posets.

Firstly, we can define ``approximations'' to derivatives from both sides.

\begin{defn}
  Let $f: A_\cplus \rightarrow B_\cpluss$ be a function. Then a \textit{sup-derivative}
  of $f$ is a function $\supderive{f}$ such that
  $$f(a \cplus \change{a}) \leq f(a) \cpluss \supderive{f}(a, \change{a})$$
  
  Similarly, a \textit{sub-derivative} of $f$ is a function $\subderive{f}$ such that 
  $$f(a \cplus \change{a}) \geq f(a) \cpluss \subderive{f}(a, \change{a})$$

  A function with a sup-derivative is sup-differentiable, and a function with a
  sub-derivative is sub-differentiable.
\end{defn}

\begin{prop}
  If $f$ is both a sup- and sub-derivative, then it is a derivative.
\end{prop}

Secondly, if the base set of a change structure is a poset, then this gives us a natural
order on the change set.

\begin{defn}[Change order]
  $\change{a} \changeOrder \change{b}$ iff for all $a \in A$ it is the case that
  $a \cplus \change{a} \leq a \cplus \change{b}$.
\end{defn}

Alternatively, the change order is the smallest order such that $\cplus$ is monotonic with
respect to its second argument.

If the change structure is extensional, then the order is antisymmetric, and a
full partial order.

Having a monotonic order on the changes is very useful.

\begin{thm}
  Let $f: A \rightarrow B$ be a function, and let $\changeOrder$ be a preorder on $\changes{B}$ such that $\cplus$ is monotonic with
  respect to it. Then let $\supderive{f}$ be a sub-derivative for $f$, and $h: A \times
  \changes{A} \rightarrow \changes{B}$ be a function such that
  $$\supderive{f} \changeOrder h$$
  Then $h$ is also a sup-derivative for $f$.

  Similarly, if $\subderive{f}$ is a sup-derivative for $f$ such that 
  $$h \changeOrder \subderive{f}$$
  Then $h$ is also a sub-derivative for $f$.
\end{thm}
\begin{proof}
  We prove the first case:
  \begin{itemize}
    \item[ ]$\supderive{f}(a, \change{a}) \changeOrder h(a, \change{a})$
    \item[$\Rightarrow$]\{ by monotonicity \}\\
      $f(a) \cplus \supderive{f}(a, \change{a}) \leq f(a) \cplus h(a, \change{a})$
    \item[$\Rightarrow$]\{ sup-derivative property \}\\
      $f(a \cplus \change{a}) \leq f(a) \cplus h(a, \change{a})$
  \end{itemize}

  The proof for the other case is symmetric.
\end{proof}

\begin{thm}[Sandwich lemma]
  \label{thm:sandwich}
  Let $\supderive{f}$ be a sup-derivative for $f$, $\subderive{f}$ be a sub-derivative for $f$, $\changeOrder$ be a preorder on $\changes{B}$ such that $\cplus$ is monotonic with
  respect to it, and $g$ be such that

  $$\supderive{f} \changeOrder g \changeOrder \subderive{f}$$

  Then $g$ is a derivative for $f$.
\end{thm}

In particular, this applies if $g$ and $h$ are themselves derivatives.

\begin{prop}
  If $f: A_\cplus \rightarrow B$ is sub-differentiable, and $A_\cplus \fineOrder
  A_\cpluss$ then $f: A_\cpluss \rightarrow B$ is sub-differentiable.
  
  Similarly, if $f: A_\cplus \rightarrow B$ is sup-differentiable, and $A_\cplus \fineOrder
  A_\cpluss$ then $f: A_\cpluss \rightarrow B$ is sup-differentiable.
\end{prop}

NEED TO GET TO THE BIT WHERE WE STICK THEM TOGETHER TO MAKE IT DIFFERENTIABLE

\subsection{Maximal and minimal derivatives}

SOME OF THIS IS WRONG

If we have a minus operator, then we can define maximal and minimal sub- and sup-derivatives.

\begin{prop}[Maximal and minimal sub- and sup-derivatives]
  \label{prop:maximalDerivatives}
  Let $A$ be a change structure and $B$ be a complete change structure, and let $f: A \rightarrow B$ be a function. Then
  $$\supderiveM{f}(a, \change{a}) = f(a \cplus \change{a}) \cminus f(a)$$
  is a minimal sup-derivative, and
  $$\subderiveM{f}(a, \change{a}) = f(a \cplus \change{a}) \cminus \bot$$
  is a maximal sub-derivative.

  TODO: should actually write a proof
\end{prop}

This then gives us a full characterisation of the derivatives on a complete
change structure.

\begin{thm}[Characterization of derivatives]
  Let $A$ be a change structure and $B$ be a complete change structure, and let
  $f: A \rightarrow B$ be a function. Then the derivatives of $f$ are precisely
  the functions $\derive{f}$ such that
  $$\subderiveM{f} \changeOrder \derive{f} \changeOrder \supderiveM{f}$$
\end{thm}
\begin{proof}
  Follows easily from \ref{thm:sandwich} and \ref{prop:maximalDerivatives}.
\end{proof}

This theorem gives us leeway when trying to pick a derivative: we can pick out the
bounds, and that tells us how much ``wiggle room'' we have. This can be helpful
because some of the intermediary functions may be much easier to compute than
others, or convenient for other reasons.

\section{Related work}

The seminal paper in this area is ~\cite{cai2014changes}. We use the notions
defined in that paper heavily, but we deviate in two regards: the use of
dependent types, and the nature of function changes.

These two issues are linked, because part of the reason that Cai et al need
dependently typed changes is in order to handle their notion of function
changes.

Our notion of function changes is different to Cai's because MARIO RANT HERE

However, even if we think that we do not need dependently typed changes for the
simple cases we have been considering, if we want to extend this formulation to
synthetic differential geometry we will need them, since the tangent space is
dependent on the point at which you are taking tangents.

Griffin?

\bibliography{paper}

\end{document}
