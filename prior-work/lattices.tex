\documentclass[english]{article}
\usepackage[T1]{fontenc}

\usepackage[latin9]{inputenc}
\usepackage{color}
\usepackage{babel}
\usepackage{mathtools}
\usepackage{amsmath}
\usepackage{stmaryrd}

\usepackage{amsthm}
\usepackage{amssymb}
\usepackage[unicode=true,pdfusetitle,
 bookmarks=true,bookmarksnumbered=false,bookmarksopen=false,
 breaklinks=true,pdfborder={0 0 0},backref=false,colorlinks=true]
 {hyperref}

\newcommand{\als}[1]{\begin{align*} #1 \end{align*}}
\newcommand{\red}[1]{{\color{red} #1}}
\newcommand{\A}[1]{\mathbf{#1}}

% Relation difference symbols
\newcommand{\relD}[1]{\Delta #1}
\newcommand{\relE}[1]{\mathcal{E} #1}
\newcommand{\relG}[1]{\Gamma #1}

% A few convenience commands
\newcommand{\nxt}[3]{\left(#1 \vee #2 \right) \wedge \neg #3}
\newcommand{\relNext}[1]{\nxt{#1}{\relD{#1}}{\relE{#1}}}
\newcommand{\BAlg}[1]{\mathcal{#1}}

% Datalog semantic bits
\newcommand{\Rel}{Rel}
\newcommand{\denote}[1]{[#1]}
\newcommand{\domP}{\mathcal{P}}
\newcommand{\consq}{\mathcal{I}}

% Exact difference symbols
\newcommand{\exactD}{d}
\newcommand{\exactE}{e}
\newcommand{\exactG}{g}

% Various forms of the positive difference operator
\newcommand{\diffL}[4]{\exactD \left(#1, #2, #3, #4\right)}
\newcommand{\diff}[2]{\exactD \left(#1, #2, \relD{#2}, \relE{#2}\right)}
\newcommand{\diffS}[1]{\exactD \left(#1\right)}

% Various forms of the negative difference operator
\newcommand{\eiffL}[4]{\exactE \left(#1, #2, #3, #4\right)}
\newcommand{\eiff}[2]{\exactE \left(#1, #2, \relD{#2}, \relE{#2}\right)}
\newcommand{\eiffS}[1]{\exactE \left(#1\right)}

% Various forms of the combined difference operator
\newcommand{\giffL}[4]{\exactG \left(#1, #2, #3, #4\right)}
\newcommand{\giff}[2]{\exactG \left(#1, #2, \relD{#2}, \relE{#2}\right)}
\newcommand{\giffS}[1]{\exactG \left(#1\right)}

% Approximate difference symbols
\newcommand{\approxD}{\delta}
\newcommand{\approxE}{\epsilon}
\newcommand{\approxG}{\gamma}

% Various forms of the positive approximate difference operator
\newcommand{\adiffL}[4]{\approxD \left(#1, #2, #3, #4\right)}
\newcommand{\adiff}[2]{\approxD \left(#1, #2, \relD{#2}, \relE{#2}\right)}
\newcommand{\adiffS}[1]{\approxD \left(#1\right)}

% Various forms of the negative approximate difference operator
\newcommand{\aeiffL}[4]{\approxE \left(#1, #2, #3, #4\right)}
\newcommand{\aeiff}[2]{\approxE \left(#1, #2, \relD{#2}, \relE{#2}\right)}
\newcommand{\aeiffS}[1]{\approxE \left(#1\right)}

% Various forms of the combined approximate difference operator
\newcommand{\agiff}[2]{\approxG \left(#1, #2, \relD{#2}, \relE{#2}\right)}
\newcommand{\agiffS}[1]{\approxG \left(#1\right)}


\makeatletter
%%%%%%%%%%%%%%%%%%%%%%%%%%%%%% Textclass specific LaTeX commands.

\theoremstyle{plain}
\newtheorem{thm}{\protect\theoremname}
\theoremstyle{remark}
\newtheorem{rem}[thm]{\protect\remarkname}

\theoremstyle{remark}
\newtheorem{claim}[thm]{\protect\claimname}

\theoremstyle{remark}
\newtheorem{corollary}[thm]{\protect\corollaryname}

\theoremstyle{definition}
\newtheorem{defn}[thm]{\protect\definitionname}

\theoremstyle{definition}
\newtheorem{prop}[thm]{\protect\propositionname}

%%%%%%%%%%%%%%%%%%%%%%%%%%%%%% User specified LaTeX commands.
\newcommand{\ra}[0]{\rightarrow}
\let\olditemize\itemize
\renewcommand{\itemize}{
  \olditemize
  \setlength{\itemsep}{1pt}
  \setlength{\parskip}{0pt}
  \setlength{\parsep}{0pt}
}

\makeatother

\providecommand{\claimname}{Claim}
\providecommand{\corollaryname}{Corollary}
\providecommand{\definitionname}{Definition}
\providecommand{\propositionname}{Proposition}
\providecommand{\remarkname}{Remark}
\providecommand{\theoremname}{Theorem}

\newcommand{\kite}[0]{\mathbin{\rotatebox[origin=c]{45}{$\boxtimes$}}}
\newcommand{\kminus}[0]{\mathbin{\rotatebox[origin=c]{-45}{$\boxslash$}}}
\newcommand{\Dia}[0]{\mathbin{\rotatebox[origin=c]{45}{$\square$}}}

\begin{document}

\title{Insert clever pun here}
\author{}

\maketitle

\section{A comment on the type of derivatives}

Michael Arntzenius' notion of derivative is really interesting, because the types seem lifted right
off DiLL (rather than Cai's formalism, even though they fit)

Given a non-monotonic function $f : A \ra B$, i.e. a monotonic function $f : \square A \ra B$
(where $\square A$ denotes the discrete order in $A$, note that this is a comonad), a derivative
is a monotonic function $\square A \times A \ra B$, i.e. a function of two arguments which is
monotonic in its second one. Furthermore, it is ``almost linear'': if it were linear, we would
expect $f'(a, da) \vee f'(a, db) = f'(a, da \vee db)$, but instead (by monotonicity) we have
$f'(a, da) \vee f'(a, db) \leq f'(a, da \vee db)$. Thus, sublinearity.

\section{Example on the real numbers}

Given a function $f$, we wish to decompose it as $f(x + a) = f(x) \oplus g(x, a)$ - in order to
make the types fit, $g$ should be monotonic in its second argument. The intuition on how to
do this is clear on a whiteboard, but the details are tricky.

To make it easier, we suppose $f$ to be differentiable (differentiable almost anywhere might
work too, but it might be trickier to prove). Now we define:
\begin{align*}
  f^+(x, dx) \triangleq \int_x^{x + dx} \text{max}(f'(z), 0) dz\\
  f^-(x, dx) \triangleq \int_x^{x + dx} \text{min}(f'(z), 0) dz\\
\end{align*}
Note that $f^+$ is monotonic and $f^-$ is anti-monotonic (both on their second argument, but not
the first). Furthermore:
\begin{align*}
  f(x + dx) &= f(x) + \int_x^{x + dx} f'(z) dz \\
  &= f(x) + \int_x^{x + dx} \text{max}(f'(z), 0) + \text{min}(f'(z), 0) dz\\
  &= f(x) + f^+(x, dx) + f^-(x, dx)
\end{align*}

Now if we define an ordering on $\mathbb{R^2}$ by
\begin{align*}
  (x, y) \sqsubseteq (x', y') \Leftrightarrow x \leq x' \wedge y \geq y'
\end{align*}
we have that the function
\begin{align*}
  f'(x, dx) \triangleq (f^+(x, dx), f^-(x, dx))
\end{align*}
is monotonic in its second argument and furthermore verifies
\begin{align*}
  f(x + dx) = f(x) \oplus f'(x, dx)
\end{align*}
where $x \oplus (y, z) \triangleq x + y + z$.

Claim: this is exactly what's going on in the case with boolean algebras.

\section{Order from change structures}

Let $(A, \Delta A, \oplus)$ be some change structure (furthermore I assume it has the structure
of a monoid action). It then induces a preorder on the underlying set $A$, which I call the
reachability order on $A$:
\begin{align*}
  a \sqsubseteq b &\Leftrightarrow \exists da \in \Delta A . a \oplus da = b
\end{align*}

In the case of discrete change structures, i.e. $a \oplus da = a$, this is the discrete order.
If the change structure has $\ominus$, then for every $a, b$ we have $a \sqsubseteq b$.

A function $f : A_\oplus \ra B_\boxplus$ is differentiable iff it is monotonic (note that from
this section it follows that functions from discrete change structures or into change structures
with $\ominus$ are always differentiable).

\subsection{Order on differences}

As in the DREAD paper, given a preorder on $A$ there is a preorder on differences:
\als{
  da \sqsubseteq db \Leftrightarrow \forall a \in A . a \oplus da \leq a \oplus db
}
If we take the previous preorder on $A$, this becomes
\als{
  da \sqsubseteq db &\Leftrightarrow \forall a \in A .\exists dc \in \Delta A
  . a \oplus da \oplus dc = a \oplus db\\
  &\Leftrightarrow \forall a \in A . \exists dc \in \Delta A
  . a \oplus (da + dc) = a \oplus db
}
Given the extensionality condition $ \forall a . (a \oplus da) \ominus a = da$ (or, equivalently,
$ \forall a\ da\ db. a \oplus da = a \oplus db \Rightarrow da = db$), this order becomes:
\als{
  da \sqsubseteq db &\Leftrightarrow \exists dc . da + dc = db
}
which is precisely the reachability order on the change structure $(\Delta A, \Delta A, +)$.
This can actually be weakened and it suffices that
$\exists a .\forall da\ db. a \oplus da = a \oplus db \Rightarrow da = db$

\subsection{Semilattices}

Given a meet-semilattice $(A, \wedge, \top)$, there is a change structure given by
\als{
  \Delta A &\triangleq A\\
  a \oplus b &\triangleq a \wedge b
}
The reachability order from this structure is precisely the opposite order of the semilattice.
To see this,
take $b \leq a$. Then $a \wedge b = b$, thus $a \oplus b = b$ hence $a \sqsubseteq b$. Conversely,
suppose $a \sqsubseteq b$. Then $a \wedge c = b$ for some $c$ therefore $b \leq a$.

Furthermore, since $\top \wedge a = a$, the previous weak extensionality condition holds and
therefore the difference order on $A$ is the same as the reachability order (and the opposite as
the lattice order).

The exact same results are true of join-semilattices, but in that case the reachability order
is the same as the order of the lattice.

\subsection{Order on semilattice derivatives}

Let $f : A \ra B$ be some $\vee$-differentiable map. The set of its derivatives is in itself
a $\vee$-lattice, with the order being defined pointwise.

\section{Order on change structures}

Given a set $A$, we can establish a partial order on its possible change structures as follows:
let $(\Delta_1 A, \oplus_1)$, $(\Delta_2 A, \oplus_2)$ be two change structures. Then we say
$(\Delta_1 A, \oplus_1) \leq (\Delta_2 A, \oplus_2)$ if and only if the identity function on
$A$ is differentiable when consider as a function from $(A, \Delta_1 A, \oplus_1)$ into
$(A, \Delta_2 A, \oplus_2)$.

This gives rise to a preorder $\A{CStr}(A)$ instead - indeed, it gives rise to a category,
which is nothing but the fibre category over $A$ for the obvious forgetful functor. This
category has some weak limits:

\begin{prop}
  The category $\A{CStr}(A)$ has a (family of) weak initial object(s), given by
  $a \oplus da = a$.
\end{prop}

\begin{prop}
  Any change structure with $\ominus$ is a weak terminal object of $\A{CStr}(A)$.
\end{prop}

We also have the following result:

\begin{prop}
  If $f : (A, \Delta A, \oplus) \ra (B, \Delta_1 B, \oplus_1)$ is differentiable and
  $\oplus_1 \leq \oplus_2$ then $f : (A, \Delta A, \oplus) \ra (B, \Delta_2 B, \oplus_2)$ is
  differentiable (a similar result is available for pre-composition).
\end{prop}

Unfortunately this ordering is not too great: there are lots of change structures that are
``the same as'' others while being manifestly ``worse'' in a sense. For example, the family of
weak initial objects. We can shave off some of these by restricting ourselves to ``non-redundant''
structures, i.e. those satisfying weak extensionality. Under this assumption, the initial
and terminal objects are unique.

\subsection{Superposition of change structures}

Given change structures (without minus) $\oplus$ and $\boxplus$ on $A$, a superposition of these,
$\circledast$ is a weak coproduct. This is always going to be essentially the same as the
change structure:
\als{
  \Delta_\star &\triangleq (\Delta_\oplus \times \Delta_\boxplus)^\star\\
  a \star (d, \delta) &\triangleq a \oplus d \boxplus \delta
}
Perhaps they are all quotients over this? The case on Boolean algebras certainly is. How do we
quantify how much better/worse they are?

\newcommand{\cast}[0]{\circledast}
The proof that these are the same is relatively simple. Since $\cast$ is a (weak) coproduct, the
identity into $\star$ is differentiable. For the converse, it suffices to map each pair
$(d, \delta)$ into $\iota_1 d + \iota_2 \delta$ (modulo shuffling some stuff around).

Furthermore, if the smallest preorder that contains $\leq_\oplus$ and $\leq_\boxplus$ is the whole
set $A$, then $\star$ has a minus and hence every function into it (and therefore into $\cast$)
is differentiable. This is precisely what happens in Boolean algebras.

Alternatively, we can show that the reachability preorder is refined by inclusion (i.e. if a
change structure is smaller than another, then its reachability preorder is finer - adjunction
between the lattice of preorders on $A$ and change structures?). Thus, the reachability preorder of
$\cast$ is at least as coarse as the finest preorder that contains $\leq_\oplus$ and $\leq_\boxplus$.

\subsubsection{On lattices}

Consider the change structures on some lattice $A$ given by $\wedge$ and $\vee$. We can define
a superposition of these as follows (note that some details are up to choice):

\begin{align*}
  \Dia A &\triangleq \Delta A \times \nabla A\\
  a \kite (da, \delta a) &\triangleq a \oplus da \boxplus \delta a\\
  b \kminus a &{\triangleq \red{(b \ominus a, b \boxminus b)}}\\
  b \kminus a &{\triangleq \red{(a \ominus a, b \boxminus a)}}\\
  \red{(da, \delta a) + (db, \delta b) }&\red{\triangleq (da + db, \delta a + \delta b)}\\
  \red{(da, \delta a) + (db, \delta b) }&\red{\triangleq
  (da \oplus db, (\delta a \oplus db) \boxplus \delta b)}\\
  (da, \delta a) + (db, \delta b) &\triangleq
  ((da \boxplus \delta b) \oplus db, (\delta a \oplus db) \boxplus \delta b)\\
\end{align*}
Big question: this works fine in boolean algebras. Can it be generalized?

\section{Subderivatives}

Suppose a change structure $\oplus$ on a set $B$ endowed with a partial order $\leq$. Given
a function $f : A \ra B$, a function $f' : A \times \Delta A \ra \Delta B$ is a subderivative
if the following inequality holds for all $a, da$: \als{
  f(a) \oplus f'(a, da) &\leq f(a \oplus da)
}
A supderivative is exactly the same thing, but the other way around. We abuse the notation
and extend the order $\leq$ to $B$-valued functions.

Given two functions $\partial_1 f, \partial_2 f : A \times \Delta A \ra \Delta B$, we define
$\partial_1 f \leq \partial_2 f$ iff for all $a, da$ we have $f(a) \oplus \partial_1f(a, da)
\leq f(a) \oplus \partial_2f(a, da)$.

Some obvious propositions:
\begin{prop}
  $f'$ is a derivative for $f$ iff it is a sub- and sup-derivative for $f$.
\end{prop}
\begin{prop}
  If the reachability order $\sqsubseteq$ refines $\leq$ (i.e. if $x \leq y$ entails
  $x \sqsubseteq y$) then every subdifferentiable function is differentiable (the same is
  not true of supderivatives).
\end{prop}
\begin{prop}
  If $\partial_1 f, \partial_2 f$ are subderivatives of $f$ and $\partial_1 f \leq g \leq
  \partial_2 f$, then so is $g$.
\end{prop}
\begin{prop}
  If $\partial f$ is a subderivative of $f$ and $f'$ is a derivative of $f$, then
  $\partial f \leq f'$.
\end{prop}

An operator $\ominus : A \times A \ra \Delta A$ is a ``subdifference'' for a change structure
if $a \oplus (b \ominus a) \leq b$. Every such operator gives rise to a subderivative operator
$\partial_\ominus$ defined as \als{
  \partial_\ominus (f)(a, da) &= f(a \oplus da) \ominus f(a)
}

\begin{defn}
  A change structure on a poset is \textbf{ascending} iff the reachability order refines the natural
  order, i.e. whenever $f(a) \leq f(a \oplus b)$ it is the case that
  $f(a) \sqsubseteq f(a \oplus b)$.
\end{defn}
\begin{defn}
  A change structure on a poset is \textbf{descending} iff every function into the poset is
  subdifferentiable.
\end{defn}
\begin{defn}
  A change structure on a poset is \textbf{precise} iff every function into the poset is
  differentiable.
\end{defn}

\begin{prop}
  If a change structure is both ascending and descending, then it is precise
\end{prop}

\section{Orderings on difference operators}

Important, do not forget: there may be a single minimal derivative even if there is no single
derivative. This has many applications.

\section{Desiderata}

It would be cool if we could get the entirety of the theory in Griffin et al. from whatever it
is we do. We can get their notion of ``minimal changes'' from ours. What is their derivative?
It's ours, hopefully. How good is our derivative?

\section{Differential equations}

\newcommand{\defeq}[0]{\triangleq}
Given a function $f : A \ra A$ we can define a kind of ``derivative'' by setting
$f(a) = \A{Id}(a) \oplus \delta (a)$ (or, if there is $\ominus$, $f \ominus \A{Id} = \delta$)
This operator satisfies a kind of chain rule: $\delta(g\circ f) = \delta(g) \circ f + \delta(f)$

Note that such a $\delta : A \ra \Delta A$ is a kind of vector field.
Given a curve $\alpha : \mathbb{N} \ra A$ we can
define the line integral as
\begin{align*}
  \int_m^n \delta d\alpha \defeq \sum_{i = m}^{n-1} \delta(\alpha(i))
\end{align*}

An integral curve for $\delta$ is precisely one which is its own line integral, i.e.
\begin{align*}
  \alpha(m + k) = \alpha(m) \oplus \int_m^{m+k}\delta d\alpha
\end{align*}
We can check that it verifies the following differential equation:
\begin{align*}
  \alpha'(n, 1) = \delta(\alpha(n))
\end{align*}


\end{document}
