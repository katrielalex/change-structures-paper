\documentclass[english]{article}

\usepackage{color}

\usepackage{amsmath}
\usepackage{amsthm}
\usepackage{amssymb}
\usepackage{mathtools}

\usepackage[unicode=true,pdfusetitle,
 bookmarks=true,bookmarksnumbered=false,bookmarksopen=false,
 breaklinks=true,pdfborder={0 0 0},backref=false,colorlinks=true]
{hyperref}

% Theorem styles 

\theoremstyle{plain}
\newtheorem{thm}{Theorem}

\theoremstyle{definition}
\newtheorem{prop}[thm]{Proposition}

\theoremstyle{remark}
\newtheorem{claim}[thm]{Claim}

\theoremstyle{remark}
\newtheorem{corollary}[thm]{Corollary}

\theoremstyle{remark}
\newtheorem{rem}[thm]{Remark}

% Definitions on their own counter

\theoremstyle{definition}
\newtheorem{defn}{Definition}

% Notation

% General
\newcommand{\defeq}{\coloneqq}
\newcommand{\cat}[1]{\mathbf{#1}}

% Change structures
\newcommand{\cplus}{\oplus}
\newcommand{\cpluss}{\boxplus}
\newcommand{\cminus}{\ominus}
\newcommand{\cstruct}[3]{(#1,#2,#3)}
\newcommand{\changes}[1]{\Delta #1}
\newcommand{\change}[1]{\delta #1}
\newcommand{\derive}[1]{#1'}

% Algebra
\newcommand{\superpose}{\circledast}

% Orders
\newcommand{\reaches}{\leq_R}

\begin{document}

\title{Something about change structures}

\author{
  Alex Eyers-Taylor\\
  Semmle Ltd.
  \and
  Michael Peyton Jones\\
  Semmle Ltd.
  \and
  Mario Alvarez Picallo\\
  University of Oxford
}

\maketitle

\section{Change structures}

\begin{defn}[Change structures]

  A change structure is defined as:

  $$\mathcal{A} \defeq \cstruct{A}{\changes{A}}{\cplus}$$

  where $A$ is a set, $\changes{A}$ is a semigroup, and $\cplus$ gives a semigroup action on $A$.

  We will call $\changes{A}$ the change set of the change structure.
\end{defn}

Elements in the change set represent changes that can be made to elements in the
base set, with the semigroup action being the operation that ``applies'' the
change. The requirement that the change set be a semigroup is convenient but in
fact inessential: given any set with an action on the base set, we can take the
free semigroup over that set to obtain a semigroup action.

Some examples of changes structures are:
\begin{itemize}
  \item $\mathbb{Z}_1 \defeq \cstruct{\mathbb{Z}}{\mathbb{N}}{+}$
  \item $\mathbb{Z}_2 \defeq \cstruct{\mathbb{Z}}{\mathbb{Z}}{+}$
\end{itemize}

Given a change structure, we have a natural notion of a derivative, following ~\cite{cai2014changes}:

\begin{defn}[Derivatives]
  A derivative of a function $f: A \rightarrow B$, where $A$ and $B$ are
  change structures with actions $\cplus$ and $\cpluss$, is a function $\derive{f}: A \times \changes{A} \rightarrow
  \changes{B}$ such that
  $$f(a \cplus \change{a}) = f(a) \cpluss \derive{f}(a, \change{a})$$

  A function which has a derivative is called differentiable.
\end{defn}

\begin{thm}[The Chain Rule]
   $$\derive{(g \circ f)}(x, \change{x}) = \derive{g}\left(f(x), \derive{f}(x, \change{x})\right)$$
\end{thm}
\begin{proof}
  \begin{itemize}
    \item[ ]$(g \circ f)(x) \cpluss \derive{g}\left(f(x), \derive{f}(x,\change{x})\right)$
    \item[ ]$(g(f(x)) \cpluss \derive{g}\left(f(x), \derive{f}(x,\change{x})\right)$
    \item[=]$g\left(f(x) \cpluss \derive{f}(x, \change{x}) \right)$
    \item[=]$g\left(f(x \cplus \change{x})\right)$
    \item[=]$(g \circ f)(x \cplus \change{x})$
  \end{itemize}
  Therefore $\derive{g}\left(f(x), \derive{f}(x, \change{x})\right)$ is a
  derivative for $(g \circ f)$.
\end{proof}

Why call this the ``chain rule''? It doesn't have quite the same structure as
the chain rule in real calculus ($\derive{(g \circ f)}(x) = (\derive{g} \circ f)
(x) \cdot \derive{f}(x)$), but it does have the same structure as the chain rule
from differential geometry ($\derive{(g \circ f)}(x, \textbf{v}) = \derive{g}
(f(x), \derive{f}(x, \textbf{v}))$), which will turn out to be a recurring connection.

\subsection{Minus operators and completeness}

Cai et al include the presence of a minus operator in their change structure
definition. 

\begin{defn}[Minus operator]
  A minus operator is a function $\cminus: A \times A \rightarrow \changes{A}$
  such that $a \cplus (b \cminus a) = b$.
\end{defn}

We have omitted minus operators from our change structure definition because
there are many interesting change structures that do not have them (for example,
$\mathbb{Z}_2$ does, but $\mathbb{Z}_1$ does not).

\begin{defn}[Completeness]
  A change structure is complete if for any $a, b \in A$, there is
  a change $\change{a} \in \changes{A}$ such that $a \cplus \change{a} = b$.
\end{defn}

\begin{prop}[Completeness equivalences]
  Let $A$ be a change structure. The following are equivalent:
  \begin{itemize}
    \item $A$ is complete.
    \item There is a minus operator on $A$.
    \item The semigroup action of the change structure on $A$ is transitive.
  \end{itemize}
\end{prop}

\section{Algebra of change structures}

\begin{defn}[Category of change structures]
  We define the category $\cat{CStruct}$ of change structures. The objects are
  change structures and the morphisms are 
\end{defn}

TODO: both of these cases depend on the underlying category (Set) having the
requisite construction. Seems like this may just be a requirement.

\begin{prop}[Products]
  Let $\mathcal{A} = \cstruct{A}{\changes{A}}{\cplus}$ and $\mathcal{B} =
  \cstruct{B}{\changes{B}}{\cpluss}$ be change structures.

  Then $\mathcal{A} \times \mathcal{B} \defeq \cstruct{A \times B}{\changes{A} \times
  \changes{B}}{\cplus \times \cpluss}$ is their categorical product.
\end{prop}

\begin{prop}[Equalizers]
  Let $A$ and $B$ be change structures, and $f, g: A \rightarrow B$ be morphisms
  between them.

  Then let $EQ(f,g)$ be the equalizer of $f$ and $g$ in $\cat{Set}$

  Then $\cstruct{EQ(f, g)}{\emptyset}{\emptyset}$ is an equalizer for $f$ and $g$
  in $\cat{CStruct}$.
\end{prop}

\begin{corollary}
  $\cat{CStruct}$ has all finite limits.
\end{corollary}

\subsection{Superpositions}

As well as combining change structures entire, we can combine two different
change structures on the same underlying set.

\begin{defn}[Superposition]
  Let $\mathcal{A}_\cplus = \cstruct{A}{\changes{A}_\cplus}{\cplus}$ and $\mathcal{A}_\cpluss =
  \cstruct{A}{\changes{A}_\cpluss}{\cpluss}$ be change structures.

  Then the superposition of $\mathcal{A}_\cplus$ and $\mathcal{A}_\cpluss$ is defined as:
  $$\mathcal{A}_\cplus \superpose \mathcal{B}_\cpluss \defeq \cstruct{A}{
    (\changes{A}_\cplus \times \changes{A}_\cpluss)^\ast}{\star}$$

  where $X^\ast$ is the set of finite sequences of $X$, and $a \star (d, e)
  \defeq (\cpluss \circ \cplus)^\ast$.
 
\end{defn}

\section{Order and change structures}

\subsection{Orders on changes}

There is a natural ordering on changes, given by reachability under the action.

\begin{defn}[Reachability order]
  $a \reaches b$ iff there is a $\change{a} \in \changes{A}$ such that $a \cplus
  \change{a} = b$.
\end{defn}

A complete change structure has a (TODO: what is the word?) reachability order;
a discrete change structure has a discrete reachability order.

\begin{prop}
  A function is differentiable iff it is monotonic with respect to the
  reachability order.
\end{prop}

\begin{corollary}
  Any function from a discrete change structure or into a complete change
  structure is differentiable.
\end{corollary}

% remove once we actually cite things
\nocite{*}
\bibliographystyle{plain}
\bibliography{paper}

\end{document}
